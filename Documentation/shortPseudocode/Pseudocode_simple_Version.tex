\documentclass{article}

\usepackage{amsmath}
\usepackage{algorithm}
\usepackage{algorithmic}
\usepackage{german}
\usepackage{color}
\title{Small Summery}
\author{Boris Prochnau}
\begin{document}
\maketitle

\section{Pseudocode}
One thing to mention is that the Pseudocode snippets are marked as "`Algorithm"', but in fact they are just functions and I don’t know yet how to change the name in this particular LaTeX environment. Also I use a "`="' instead of "`$\leftarrow$"' because I think it provides better readability.\\
I separated this section in 3 parts, each one is dedicated to one of the functions used in EvolutionStep() below. It is possible to read \\
We will encounter 2 classes called "`Trait"' and "`Events"'.\\

\begin{minipage}[t]{0.5\textwidth}
Attributes in a Trait Object:\\
	\textcolor[rgb]{0,0,0.55}{
	\textbf{class Trait}
	\begin{itemize}
		\item Members
		\item BirthRate
		\item DeathRate
		\item TotalBirthRate
		\item TotalDeathRate
		\item TotalTraitRate
		\item TotalEventRate - [static]
		\item CompDeathRate[i][j] - [static]
		\item Mutation - [static]
	\end{itemize}
	}
\end{minipage}
\begin{minipage}[t]{0.5\textwidth}
Attributes in a Events Object:\\
	\textcolor[rgb]{0,0,0.55}{\textbf{class Events}
	\begin{itemize}
		\item EventTimes[i]
		\item ChosenTrait[i]
		\item isBirth[i]
	\end{itemize}
	}
\end{minipage}


\begin{algorithm}[H]
	\caption{EvolutionStep()}
	\begin{algorithmic}[1]
		\REQUIRE -
		\ENSURE{A full evolution Step happened}
		\STATE calculateEventRates();
		\STATE sampleEventTime();
		\STATE changeATrait();
	\end{algorithmic}
\end{algorithm}
This function does a full evolution step.

\subsection{Calculating total-event rates}

\begin{algorithm}[H]
	\caption{calculateEventRates()}
	\begin{algorithmic}[1]
		\REQUIRE -
		\ENSURE{All (total)Rates will be set}
		\FOR{i=0 \TO n-1}
			\STATE calculateTotalDeathRateOf(i)
		\ENDFOR
		\STATE calculateTotalBirthRates(0);
		\STATE calculateTotalEventRate();
	\end{algorithmic}
\end{algorithm}

\begin{algorithm}[H]
	\caption{calculateTotalDeathRateOf(TraitIndex: i)}
	\begin{algorithmic}[1]
		\REQUIRE int i
		\ENSURE{Total deathrate of Trait "`i"' will be set}
		\STATE Trait[i].TotalDeathRate = 0;
		\STATE addTotalIntrinsicDeathRateOf(i);
		\STATE addTotalCompetitionDeathRateOf(i);
	\end{algorithmic}
\end{algorithm}

\begin{algorithm}[H]
	\caption{addTotalIntrinsicDeathRateOf(TraitIndex: i)}
	\begin{algorithmic}[1]
		\STATE Trait[i].TotalDeathRate = (Trait[i].DeathRate) $\cdot$ (Trait[i].Members)
	\end{algorithmic}
\end{algorithm}
\begin{algorithm}[H]
	\caption{addTotalCompetitionDeathRateOf(TraitIndex: i)}
	\begin{algorithmic}[1]
		\STATE double externalDeath = 0;
		\FOR{j=0 \TO n-1}
			\STATE externalDeath += (Trait.CompDeathRate[i,j])$\cdot$(Trait[j].Members);
		\ENDFOR
		\STATE externalDeath *= Trait[i].Members;
		\STATE Trait[i].TotalDeathRate += externalDeath;
	\end{algorithmic}
\end{algorithm}

\begin{algorithm}[H]
	\caption{calculateTotalBirthRates(StartIndex: i)}
	\begin{algorithmic}[1]
		\REQUIRE int i
		\ENSURE{Total birthrate of Trait "`i"' will be set (recursively)}
		\STATE Trait[i].TotalBirthRate = (Trait[i].Members)$\cdot$(Trait[i].BirthRate)
		\IF{$i < n-1$}
			\STATE calculateTotalBirthRates(i+1)
			\STATE Trait[i].TotalBirthRate += $\frac{Trait.Mutation}{2}\cdot Trait[i+1].BirthRate \cdot Members $
		\ENDIF
		\IF{$i > 0$}
			\STATE Trait[i].TotalBirthRate += $\frac{Trait.Mutation}{2}\cdot Trait[i-1].BirthRate \cdot Members $
		\ENDIF
	\end{algorithmic}
\end{algorithm}
In Algorithm 6 in line 3 is used recursion, because this improves the calculation speed a lot, although it slightly makes code less intuitive.

\begin{algorithm}[H]
	\caption{calculateTotalEventRate()}
	\begin{algorithmic}[1]
		\REQUIRE -
		\ENSURE{Current Totaleventrate is set}
		\STATE Trait.TotalEventRate = 0;
		\FOR{i=0 \TO n-1}
			\STATE Trait[i].TotalTraitRate = Trait[i].TotalBirthRate \\ 
			\noindent\hspace*{36mm}+ Trait[i].TotalDeathRate;
			\STATE Trait.TotalEventRate += Trait[i].TotalTraitRate;
		\ENDFOR
	\end{algorithmic}
\end{algorithm}

\subsection{Sampling the next event-time}
Here will appear a, not yet mentioned, object that will not be explained further, called Dice. The Dice Object will provide a uniform or exponential random Variable.\\
\begin{algorithm}[H]
	\caption{sampleEventTime()}
	\begin{algorithmic}[1]
		\REQUIRE -
		\ENSURE{First ringing Eventclock has been sampled}
		\STATE double Parameter = Trait.TotalEventRate;
		\STATE double newEvent = this.Dice.RollExpDice(Parameter);
		\STATE Events.EventTimes.push(newEvent);
	\end{algorithmic}
\end{algorithm}
Here we use Dice.RollExpDice($\lambda$) to get $X\sim exp(\lambda)$. The same is possible for  Dice.RollUnifDice($\lambda$) to get $X\sim Unif[0,\lambda]$.\\

\subsection{Changing a trait}

\begin{algorithm}[H]
	\caption{changeATrait()}
	\begin{algorithmic}[1]
		\REQUIRE -
		\ENSURE{make a change to the Population with current Parameters}
		\STATE choseTraitToChange();
		\STATE choseEventType();
		\STATE executeEventTypeOnTrait();
	\end{algorithmic}
\end{algorithm}

\begin{algorithm}[H]
	\caption{choseTraitToChange()}
	\begin{algorithmic}[1]
		\REQUIRE -
		\ENSURE{Trait is chosen for changing}
		\STATE double Parameter = Trait.TotalEventRate;
		\STATE double HittenTrait = Dice.rollUnif(Parameter);
		\FOR{i = 0 \TO n-1}
			\IF{HittenTrait $\le$ Trait[i].TotalTraitRate}
				\STATE Events.ChosenTrait.push(i);
				\STATE break;
			\ENDIF
			\STATE HittenTrait -= Trait[i].TotalTraitRate;
		\ENDFOR
	\end{algorithmic}
\end{algorithm}

\begin{algorithm}[H]
	\caption{choseEventType()}
	\begin{algorithmic}[1]
		\REQUIRE -
		\ENSURE{Decision for Birth or Death is made}
		\STATE int i = Events.ChosenTrait.lastentry();
		\STATE double EventType = Dice.rollUnif(Trait[i].TotalTraitRate);
		\IF{EventType $\le$ Trait[i].TotalBirthRate}
			\STATE Events.isBirth.push(true);
		\ELSE
			\STATE Events.isBirth.push(false);
		\ENDIF
	\end{algorithmic}
\end{algorithm}

\begin{algorithm}[H]
	\caption{executeEventTypeOnTrait()}
	\begin{algorithmic}[1]
		\REQUIRE 2 other alg (later)
		\ENSURE{Chosen event will occur on chosen trait}
		\IF{isBirth.lastentry}
			\STATE Trait[Event.ChosenTrait.lastentry()].Members += 1;
		\ELSE
			\STATE Trait[Event.ChosenTrait.lastentry()].Members -= 1;
		\ENDIF
	\end{algorithmic}
\end{algorithm}
\listofalgorithms
\end{document}