\documentclass{beamer}

\usepackage[utf8]{inputenc}
\usepackage[ngerman]{babel}
\usepackage[T1]{fontenc}

\usepackage{amsmath,amssymb,amsfonts,amsthm,mathtools}


\mode<presentation>
{
%  \setbeamertemplate{background canvas}[vertical shading][bottom=orange!13,top=blue!13]
  \setbeamertemplate{navigation symbols}{}
    \usetheme{Dresden}
    %alternativ: nichts Dresden/malmoe/Boadilla
   \setbeamercovered{transparent}
   \useoutertheme{infolines}
}


\title[abc] % (optional, nur bei langen Titeln noetig)
{aaa}
\subtitle{bbb}
\author[B.Prochnau] % (optional, nur bei vielen Autoren)
{Boris Prochnau}
%\institute{Research Institute for Discrete Mathematics Bonn}
\institute[University of Bonn]{Institute for Applied Mathematics\\University of Bonn}

\date {\today}


\usepackage{paralist}

\usepackage[algoruled,algo2e,vlined,titlenotnumbered,german]{algorithm2e}

\usepackage{tikz}

\setbeamercovered{invisible}

% ä Ä ö Ö ü Ü
\begin{document}
%
% ****************************************************************************************************
%					preliminaries
% ****************************************************************************************************
%

\begin{frame}
  \titlepage
\end{frame}


% \section<presentation>*{Outline}

\begin{frame}
  \frametitle{Übersicht}
  \tableofcontents
  % Die Option [pausesections] koennte nuetzlich sein.
\end{frame}

\section{Model}
\begin{frame}
\frametitle{Titel}
\end{frame}
	\subsection{Normalisierung}
	\begin{frame}
	\frametitle{Titel}
	\end{frame}
	\subsection{Equilibrium}
	\begin{frame}
	\frametitle{Titel}
	\end{frame}
\section{Algorithmus}
\begin{frame}
\frametitle{Titel}
\end{frame}
\section{Simulation}
	\subsection{Aufgabenteilung und Flexibilität}
	\subsection{Layout}
\section{Korrektheit der Implementation}
\section{TSS - Prozesse}
\section{Schwierigkeiten}



\end{document}
% ä Ä ö Ö ü Ü