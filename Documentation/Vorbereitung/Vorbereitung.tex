\documentclass[11pt,a4paper,german]{article}
\usepackage[utf8]{inputenc}
\usepackage{amsmath}
\usepackage{amsfonts}
\usepackage{amssymb}
\usepackage{makeidx}
\usepackage{graphicx}
\usepackage[ngerman]{babel}

\author{Boris Prochnau}
\title{Vorbereitung zur Therapie}
\begin{document}
\maketitle

\section{Innere Ziele: m"oglichst bildlich formuliert}
\begin{itemize}
	\item Pr"ufungen sollten f"ur mich keine Quelle der Angst sondern eine Gelegenheit seien auf die ich mich freue. Sie sollten daf"ur stehen dass ich die Chance habe all das Wissen tats"achlich auf einer tiefen Ebene zu erlernen und eine entsprechend realistische Chance auf eine gute Note habe!
	\item Ich m"ochte keine Referenz mehr zu meinen fr"uheren versagten Leistungen machen m"ussen. Ein runder Abschluss zur Vergangenheit als sich nicht zwingend wiederholende Geschichte.
	\item Statt frustriert w"urde ich gerne mit freudiger Erwartung die Pr"ufungsvorbereitung angehen. Und wenn machbar auch weniger Vergn"ugungssucht, bzw. keine Hilflosigkeit mehr gegen eigene subtile Ausweichstrategien dem Lernverhalten gegen"uber.
	\item Ich sollte keine Angst mehr davor haben meine Konzentration w"ahrend fachlichen Diskussionen zu verlieren und selbstbewusst reagieren wenn etwas nicht klar ist.
\end{itemize}
\subsection{Ziel erreicht: Was w"urde sich "andern?}
Hier ist eine Liste von "Anderungen die ich auf kurze und lange Sicht erwarte. Darunter sind "Anderungen enthalten die ich aus den neu gewonnenen M"oglichkeiten und dem damit einhergehenden Selbstbewusstsein selber weiterentwickeln m"ochte. Redundanz wird hier wissentlich nicht weggestrichen um bildliches Gewicht nicht zu verf"alschen.
\begin{itemize}
	\item Wenn ich Freizeit genie"se, dann mit der Freiheit keine Schuldgef"uhle haben zu m"ussen. Ich habe Arbeit geleistet und der erwartete Erfolg hat sich oder wird sich noch einstellen.
	\item Ich h"atte keine Angst mehr vor dem Lernen und der Vorbereitung. Versagen stellt sich nur ein wenn ich es zulasse!
	\item Ich empfinde Freude am lernen und bin nicht immer damit besch"aftigt der Vorstellung des Versagens auszuweichen. 
	\item Ich komme nicht mehr in die Situation mich unbewusst vor dem Lernen zu dr"ucken. Ich wei"s dass Zielstrebigkeit und Aktivit"at mehr Erf"ullung als Frust bereit h"alt! Mit beidem geht Freude "uber meine erbrachte Leistung einher.
	\item Vorbereitungen und Gelerntes bleiben gut erhalten. Auch kurz vor Pr"ufungen habe ich noch meine volle Leistungsf"ahigkeit und glaube daran eine gute Note erreichen zu k"onnen!
	\item Ich w"are zufriedener mit mir selbst, und das erleichtert mir die Entwicklung meines Geistes aktiv zu beeinflussen.
	\item Statt meiner Misserfolge bestimmen nun meine Erfolge mein Selbstwert und mein Denken! Auf diese Weise suche ich nicht mehr st"andig nach dem Versagen und bin weniger Selbstkritisch.
\end{itemize}
\subsection{Ziel erreicht: Wie w"urde sich das anf"uhlen?}
\begin{itemize}
	\item Wenn ich eine gute Note erreicht habe, werde ich auch das Gef"uhl haben diese verdient zu haben! Dar"uber hinaus sind mir auch die Vielz"ahligen kleiner und gro"ser praktischer Erfolge bewusst, die st"andig beim Arbeiten unterkommen.
\end{itemize}
\subsection{Ziel erreicht: Was w"urde sich an meiner zwischenmenschlichen Beziehung "andern? Arbeit, Freizeit etc.}
\begin{itemize}
	\item Ich habe tiefes Vertrauen in mich und dass ich alles erlernen kann. Es gibt keinen Grund mehr sich vor einer Blamage in Gespr"achen zu f"urchten, denn Verwirrung kann ich stets irgendwann aufl"osen ohne daran scheitern zu m"ussen (d.h. Seelenruhe und Aktivit"at).
	\item Ich w"urde nicht mehr all zu oft von Selbstzweifeln geplagt die ich abwehren m"usste und ich komme nicht mehr in die Situation mich vor Pr"ufungen vor anderen klein zu machen.	
\end{itemize}

\section{Was habe ich schon alles getan?}
\begin{itemize}
	\item Buddhismus, Meditation, Buddhas Lehren.
	\item Buch: Erfolgreich Lernen
	\item Hypnose im K"oln Bonn Institut bei Herr Norbert Schick
\end{itemize}

\subsection{Erfolge}
	\begin{itemize}
		\item Wenn ich stressfrei in den Zustand komme dass ich meine F"ahigkeiten nutze wie sie sind und nicht wie sie sein sollten und sie ohne Defizit erlebe, habe ich teilweise sehr gute Leistungen erbracht. Dies geschieht in der Regel in einer Phase emotionaler Ausgeglichenheit und ist noch nicht in Pr"ufungssituationen aufgetreten.
	\end{itemize}


\clearpage
\section{Eigene Erg"anzung}
	Meine Sichtweise verstehen ist chaotisch weil. Buddhismus, h"aufiges verwerfen aller Konzepte um neuen und alten Platz zu machen. H"alt Geist flexibel.
	Probleme: Selbstbewusstsein, pr"agende Vergangenheit, "Angste. Vielleicht habe ich kein Problem und bilde mir alles nur ein?
\section{Probleme}
\begin{itemize}
%	\item Defizitorientierung: Kein wirklicher Glaube daran gute Noten verdient zu haben, wenn sie mal erreicht sind. Oft senkt sich meine Notenerwartung auch noch kurz vor den Pr"ufungen.
	\item Immer schlechte Noten und auf Defizit getrimmt worden. Keinen wirklichen Vater, hat gefordert aber nichts beigetragen.
	\item Wenn es im Gespr"ach mit anderen "{}guten"{} Mathematikern um Mathematik geht, dann schw"achelt das Selbstbewusstsein. Und vor Pr"ufern blamiere ich mich gerne mit Stumpfsinnigkeit und gestammel statt selbstbewusstem Auftreten.
%	\item In Traumatischen Situationen (unter Stresstests) mangelt manchmal das Ged"achtnis und besonders die Konzentration. Konzentration bedeutet alles f"ur Mathematiker und meine Art zu leben.
%	\item Oft schon in der Vorbereitungsphase von Selbstzweifel geplagt. Ich wehre sie meist ab, jedoch manchmal werde ich nachl"assig.
%	\item Bereits beim Lernen resultieren die obigen Ph"anomene in Angst vor der zu erbringenden Leistung und vor der erwarteten Note.
	\item Evtl. aus dem obigen Grund versuche ich mich oftmals vor dem Lernen zu dr"ucken. Ich weiche subtil meinen Pflichten aus.
\end{itemize}

\section{M"ogliche Hindernisse der Therapie}
\begin{itemize}
	\item Wingwave besteht aus EMDR, Myostatik-/O-Ringtest und NLP. Leider nehme ich die Kritik am NLP ernst und hoffe objektive Anwendung der Methoden aus dem NLP (Zitat: "{}Zudem nutze ich Inhalte aus..."{} wei"st hoffentlich auf eine Sachgem"a"se Anwendung hin).
	\item Wingwave ist ein "{} Kurzzeit-Coaching-Konzept "{}. Ist damit gemeint schnell aber nicht auf Dauer?
	\item Gedankenautobahn im Kopf: Oft reicht die Kleinste Ursache um eine Gedankenautobahn in meinem Kopf aufzubauen die konzentriertes Arbeiten schwer gestaltet.
	\item Vertrauen in den Therapeuten: Ich war schon einmal in einer Hypnosetherapie bei Norbert Schick. Doch schon ein wenig vor der Hypnose und auch w"ahrend konnte ich den unterbewussten Gedanken nicht verscheuchen dass es mir schwer f"allt diesen Menschen zu respektieren. Denn trotz seiner sicher sehr wirkungsvollen Methoden sind seine Weltansichten und seine fachliche Kompetenz f"ur mich Fragw"urdig gewesen.
\end{itemize}
\end{document}