\documentclass[11pt,a4paper,german]{article}
\usepackage[utf8]{inputenc}
\usepackage{amsmath}
\usepackage{amsfonts}
\usepackage{amssymb}
\usepackage{makeidx}
\usepackage{graphicx}
\usepackage[ngerman]{babel}

\author{Boris Prochnau}
\title{Vorbereitung zur Therapie}
\begin{document}
\maketitle

\section{Innere Ziele: m"oglichst bildlich formuliert}
\begin{itemize}
	\item Pr"ufungen sollten f"ur mich keine Quelle der Angst sondern eine Gelegenheit seien auf die ich mich freue. Sie sollten daf"ur stehen dass ich die Chance habe all das Wissen tats"achlich auf einer tiefen Ebene zu erlernen und eine entsprechend realistische Chance auf eine gute Note habe!
	\item Ich m"ochte keine Referenz mehr zu meinen fr"uheren versagten Leistungen machen wollen oder Mitgef"uhl anderer erhoffen. Also die Vergangenheit als sich nicht zwingend wiederholende Geschichte sehen.
	\item Statt frustriert w"urde ich gerne mit freudiger Erwartung die Pr"ufungsvorbereitung angehen statt subtilen Ausweichstrategien nachzugehen.
	\item Ich sollte keine Angst mehr davor haben meine Konzentration w"ahrend fachlichen Diskussionen zu verlieren und selbstbewusst reagieren wenn etwas nicht klar ist.
\end{itemize}
\subsection{Ziel erreicht: Was w"urde sich "andern?}
\begin{itemize}
	\item Wenn ich Freizeit genie"se, dann mit der Freiheit keine Schuldgef"uhle haben zu m"ussen. Ich habe Arbeit geleistet und der erwartete Erfolg hat sich oder wird sich noch einstellen.
	\item Ich empfinde Freude am lernen und bin nicht immer damit besch"aftigt der Vorstellung des Versagens auszuweichen oder habe Angst davor. 
	\item Ich komme nicht mehr in die Situation mich unbewusst vor dem Lernen zu dr"ucken. Ich wei"s dass Zielstrebigkeit und Aktivit"at eher Erf"ullung als Frust bereit h"alt! Mit beidem geht Freude "uber meine erbrachte Leistung einher.
	\item Vorbereitungen und Gelerntes bleiben gut erhalten. Auch kurz vor Pr"ufungen habe ich noch meine volle Leistungsf"ahigkeit und glaube daran eine gute Note erreichen zu k"onnen!
	\item Ich w"are zufriedener mit mir selbst, und das erleichtert mir die Entwicklung meines Geistes aktiv zu beeinflussen.
	\item Statt meiner Misserfolge bestimmen nun meine Erfolge mein Selbstwert und mein Denken! Auf diese Weise suche ich nicht mehr st"andig nach dem Versagen und bin weniger Selbstkritisch.
\end{itemize}

\subsection{Ziel erreicht: Wie w"urde sich das anf"uhlen?}
\begin{itemize}
	\item Wenn ich eine gute Note erreicht habe, werde ich auch das Gef"uhl haben diese verdient zu haben! 
	\item Die Vielz"ahligen kleinen und gro"sen Erfolge die st"andig beim Arbeiten aufkommen werden mir bewusst und motivieren mich alles zu tun was anliegt!
	\item Ich h"atte das Gef"uhl viel Erreichen zu k"onnen wenn ich etwas Anfange.
	\item Es f"uhlt sich gut an zur"uck auf meine Erfolge und besonders mein Lernverhalten zu blicken.
\end{itemize}

\subsection{Ziel erreicht: Was w"urde sich an meiner zwischenmenschlichen Beziehung "andern? Arbeit, Freizeit etc.}
\begin{itemize}
	\item Ich habe tiefes Vertrauen in mich und dass ich alles erlernen kann. Es gibt keinen Grund mehr sich vor einer Blamage in Gespr"achen zu f"urchten, denn Verwirrung kann ich stets irgendwann aufl"osen ohne daran scheitern zu m"ussen (d.h. Seelenruhe und Aktivit"at).
	\item Ich w"urde nicht mehr all zu oft von Selbstzweifeln geplagt die ich abwehren m"usste.
	\item Ich komme nicht mehr in die Situation mich vor Pr"ufungen vor anderen klein zu machen oder Mitgefühl zu suchen.
	\item Ich halte mich selbst trete diszipliniert und Zielstrebig auf.
\end{itemize}

\section{Was habe ich schon alles getan?}
\begin{itemize}
	\item Aus vergangener Unzufriedenheit erwuchs der Wunsch ein besserer Mensch zu werden und mehr Kontrolle "uber mich zu gewinnen. Das gelang mir sehr gut nachdem ich einer buddhistischen Gruppe beigetreten bin um mehr "uber mich und meinen Geist zu lernen. Das erm"oglichte mir neue Sichtweisen zu entwickeln, die mir mein Leben und Studieren sehr vereinfacht haben.
	\item Meditation sorgt meistens f"ur Ausgeglichenheit die sich gemischt mit der Erfahrung der Welt zu einer Charaktereigenschaft ausbilden. Jedoch fehlt mir oft gerade dort Erfahrung wo ich sie dringend gebrauchen k"onnte (Pr"ufungen).
	\item Um mein Lernverhalten besser verstehen zu k"onnen habe ich mir das Buch "{}Erfolgreich Lernen"{} von Monika Löhle \& Eberhardt Hofmann.
	\item Einmalige Hypnose im K"oln Bonn Institut bei Herr Norbert Schick. Leider etwas schwammige Formulierung des Wunsches und mangelndes Vertrauen zum Hypnotiseur.
\end{itemize}

\subsection{Erfolge}
	\begin{itemize}
		\item Dank Buddhismus finde ich immer wieder Wege aus Frust oder geringem Selbstwert.
		\item Das Buch hat mich motiviert mir manchmal auch bewusst meine Erfolge vor Augen zu halten und mich daran zu erfreuen. 
		\item Die Hypnose hat wahrscheinlich doch etwas bewirkt, denn mein Lernverhalten hat sich intensiv ge"andert. Durch was ist aber nicht ganz klar. Ich nehme mein Problem immer ernster und gehe immer entschlossener gegen diese Charakterschw"ache vor.
		\item Wenn ich stressfrei in den Zustand komme dass ich meine F"ahigkeiten nutze wie sie sind und nicht wie sie sein sollten und sie ohne Defizit erlebe, kann ich oft au"serordentlich gute Leistungen erbringen. Dies geschieht in der Regel in einer Phase emotionaler Ausgeglichenheit und ist noch nicht in echten Pr"ufungssituationen aufgetreten.
	\end{itemize}


\clearpage
\section{Eigene Erg"anzung}
	\begin{itemize}
		\item Meine Sichtweise zu verstehen ist chaotisch weil der Buddhismus beinhaltet h"aufiges verwerfen aller Konzepte um neuen und alten Platz zu machen. Das h"alt Geist flexibel.
		\item Ich bin mir nicht sicher was genau die Ursache meines Problems ist, Selbstbewusstsein, pr"agende Vergangenheit oder irgendwelche Versagens"angste. Vielleicht bilde ich mir meine Probleme nur ein und ich sollte sie nur mit mehr Ausdauer und Entschlossenheit bek"ampfen. Jedoch fallen mir viele mit scheinbar weniger Ausdauer oder Entschlossenheit auf, die keine Spur dieser Probleme aufweisen.
	\end{itemize}
	
%\section{Probleme}
%\begin{itemize}
%	\item Defizitorientierung: Kein wirklicher Glaube daran gute Noten verdient zu haben, wenn sie mal erreicht sind. Oft senkt sich meine Notenerwartung auch noch kurz vor den Pr"ufungen.
%	\item Immer schlechte Noten und auf Defizit getrimmt worden. Keinen wirklichen Vater, hat gefordert aber nichts beigetragen.
%	\item Wenn es im Gespr"ach mit anderen "{}guten"{} Mathematikern um Mathematik geht, dann schw"achelt das Selbstbewusstsein. Und vor Pr"ufern blamiere ich mich gerne mit Stumpfsinnigkeit und gestammel statt selbstbewusstem Auftreten.
%	\item In Traumatischen Situationen (unter Stresstests) mangelt manchmal das Ged"achtnis und besonders die Konzentration. Konzentration bedeutet alles f"ur Mathematiker und meine Art zu leben.
%	\item Oft schon in der Vorbereitungsphase von Selbstzweifel geplagt. Ich wehre sie meist ab, jedoch manchmal werde ich nachl"assig.
%	\item Bereits beim Lernen resultieren die obigen Ph"anomene in Angst vor der zu erbringenden Leistung und vor der erwarteten Note.
%\end{itemize}

\section{M"ogliche Hindernisse der Therapie}
\begin{itemize}
	\item Wingwave besteht aus EMDR, Myostatik-/O-Ringtest und NLP. Leider nehme ich die Kritik am NLP ernst und hoffe objektive Anwendung der Methoden aus dem NLP (Zitat: "{}Zudem nutze ich Inhalte aus..."{} wei"st hoffentlich auf eine Sachgem"a"se Anwendung hin).
	\item Wingwave ist ein "{} Kurzzeit-Coaching-Konzept "{}. Ist damit gemeint schnell aber nicht auf Dauer?
	\item Gedankenautobahn im Kopf: Oft reicht die Kleinste Ursache um eine Gedankenautobahn in meinem Kopf aufzubauen die manchmal konzentriertes Arbeiten schwer gestaltet. Ursache daf"ur ist wahrscheinlich mangelndes Vertrauen.
	\item Vertrauen in den Therapeuten: Ich war schon einmal in einer Hypnosetherapie bei Norbert Schick. Doch schon ein wenig vor der Hypnose und auch w"ahrend konnte ich den unterbewussten Gedanken nicht verscheuchen dass es mir schwer f"allt diesen Menschen zu respektieren. Denn trotz seiner sicher sehr wirkungsvollen Methoden sind seine Weltansichten und seine fachliche Kompetenz f"ur mich fragw"urdig gewesen.
\end{itemize}
\end{document}