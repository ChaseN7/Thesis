
\documentclass[11pt, a4paper, german]{article}
\usepackage[utf8]{inputenc}
\usepackage{amsmath}
\usepackage{amsfonts}
\usepackage{amssymb}
\usepackage{algorithm}
\usepackage{algorithmic}
\usepackage{makeidx}
\usepackage{graphicx}
\usepackage{tabularx}
\usepackage{bbm}
\usepackage[ngerman]{babel}
\usepackage{BA_Titelseite}
\bibliographystyle{unsrt}

%Namen des Verfassers der Arbeit
\authornew{Boris Prochnau}

%Geburtsdatum des Verfassers
\geburtsdatum{22. Dezember 1989}
%Gebortsort des Verfassers
\geburtsort{Tartu}
%Datum der Abgabe der Arbeit
\date{\today}

%Name des Betreuers
% z.B.: Prof. Dr. Peter Koepke
\betreuer{Betreuer: Prof. Dr. Anton Bovier}
%Name des Instituts an dem der Betreuer der Arbeit tätig ist.
%z.B.: Mathematisches Institut
\institut{Institut für Angewandte Mathematik}
%Titel der Bachelorarbeit
\title{Simulation normalisierter BPDL Prozesse}
%Do not change!
\ausarbeitungstyp{Bachelorarbeit Mathematik}



\begin{document}

\maketitle
\tableofcontents

\clearpage


\section{Einleitung}
Das Ziel dieser Bachelorarbeit ist es eine Programm zu entwickeln welches den zeitlichen Verlauf von sogenannten normalisierten BPDL Prozessen graphisch darstellt.\\
Diese BPDL (von Bolker, Pacala, Dieckmann und Law) Prozesse basieren auf einem stochastischen Populationsmodell, welches im n"achsten Kapitel n"aher vorgestellt wird. \\

Dem Programm sollte es m"oglich sein die Parameter einer Population festlegen zu k"onnen und die Entwicklung dieser Population schlie"slich zeitlich verfolgen zu k"onnen. Auf diese Weise soll man beobachten k"onnen ob sich ein Merkmal gegen"uber einem anderen Durchsetzten kann oder sich ein stabiler Zustand einpendelt. Insbesondere kann der Einfluss der Normalisierung auf das vorzeitige Aussterben eines Prozesses deutlich dargestellt werden.\\

Alle Simulationen basieren auf einem Modell, dass jedes Lebewesen einer Population (z.B. Pflanzen) ein bestimmtes Merkmal tr"agt. Diese Merkmale werden durch Wettbewerb zu jeder existenten Gruppe, Geburten und Todesraten klassifiziert. Schlie"slich ist es jedoch die Entwicklung der Population und nicht der Individuen die simuliert werden soll, weshalb man im simulierten Prozess zwar den Tod und die Geburt von Individuen verfolgen kann, aber nicht die Entwicklung spezieller Individuen. Der "Ubergang zu dieser Sichtweise wird n"aher im 2. Kapitel beschrieben und kr"ont in der Konvergenz zu einer deterministischen Funktion.\\

Um zu Pr"ufen ob und wie schnell der Prozess gegen seinen Stabilen Zustand konvergiert, werden in der Simulation auch stabile Zust"ande f"ur diverse Situationen automatisch dazu gestellt. Anhand dieser k"onnen individuelle und dynamische Abbruchkriterien formuliert werden.\\

Schlie"slich werde ich noch kurz die TSS Prozesse und ein weiteres Programm vorstellen, welches die bisher betrachteten BPDL Prozesse erweitert auf TSS-Prozesse. Dabei sollen nicht nur das besonders interessante Verhalten vom Wechsel des dominanten Merkmals simuliert werden, sondern es wird eine verbesserte Laufzeit durch Interpolation vorgestellt die eine effiziente Simulation trotz sehr gro"ser Zeit und besonders pr"azise Betrachtung von Aktionsreichen Gebieten anbietet.


%Dabei ist das Interesse besonders bei normalisierten BPDL Prozessen und schließlich auch bei TSS (trait subsitution sequence) Prozessen.\\
%Dabei reicht es nicht einfach eine Implementierung zu machen, denn durch die Zufallseigenschaft lässt sich nicht so einfach die Korrektheit verifizieren. Gerade bei TSS Prozessen ist die Anzahl der Mutationen und deren Abstände und Invasionschancen entscheidend.\\
%Zu diesem Zweck wird ein Teil der Arbeit diese Problemstellung behandeln.

\section{Modell}
Das verwendete Model lehnt sich an das Model aus \cite{Champagnat20061127} an. Dieses nutzt die drei grundlegenden Mechanismen von Darwins Evolutionslehre: Vererbung, Variation (Mutationen) und Selektion durch Wettbewerb um eine Menge von Merkmalen f"ur Individuen zu beschreiben. Diese bestimmen die F"ahigkeit des Individuums zu "uberleben und sich fortzupflanzen.\\
Jedoch wurden für meine Simulation einige kleine Änderungen gewünscht die im Anschluss erl"autert werden. \\
	\subsection{Grundlagen}
	Sei $ X $ der Raum der Merkmale. Jedes Individuum hat genau ein solches Merkmal $ i \in X $. Der Einfachheit halber sei X eine Indexmenge: $ X = \{1,\dots, n\} $ repräsentativ für eine Durchzählung der Merkmale und im folgenden seien $ i,j \in X $ zwei solche Merkmale. Au"serdem gilt:
	\begin{itemize}
		\item Jedes Individuum kann sich asexuell fortpflanzen oder sterben.
		\item Fortpflanzungs- und Todeszeitpunkte k"onnen durch sogenannte exponentielle Uhren (wie in \cite[S. 3]{fournier2004microscopic}) beschrieben werden. Diese Uhren haben exponentiell verteilte Weckzeiten. Durch die Gedächtnislosigkeit der Exponentialverteilung, können alle Uhren nach dem ersten Klingeln neu gestellt werden. 
		\item Die Fortpflanzung eines Individuums aus $ i $ kann jedoch auch in der Geburt eines Individuums in $ j $ resultieren.
	\end{itemize}
	Sp"ater wird deutlich dass die Zur"uckstellbarkeit der Uhren entscheiden ist um die Sichtweise von der Ebene des Individuums auf die der gesamten Population zu heben.\\
	Diese Todes und Fortpflanzungs- Ereignisse eines Individuums haben feste Raten die das dazugeh"orige Merkmal beschreiben.\\
	
	\begin{tabular}{r p{26em}}
		$ b(i) $: & Ist die Geburtenraten durch ein Individuum mit Merkmal $ i $.\\
		$ d(i) $: & Ist die natürliche Todesrate.\\
		$ c(i, j) $: & Ist die Todesrate durch Wettbewerb zwischen Individuen mit Merkmal i und j.\\
		$ \mu $: & Ist die Mutationswahrscheinlichkeit "{}auf die Nachbarn"{} mit je $ \frac{\mu}{2} $ pro Nachbar. \\
	\end{tabular}\\

	Schlie"slich lassen sich durch Superposition z.B. die beiden Todesraten zu einer gemeinsamen Todesrate zusammenfassen oder die arteigene Geburtenrate abtrennen.\\
	
	\begin{tabular}{ r p{18em} }
		$ b(i) \cdot (1 - \mu) $: & ist die arteigene Geburtenrate eines Individuums mit Merkmal $ i $\\
		$ d(i) + \sum_{j=1}^{N_t} c(i, j) $: & ist die gesamte Todesrate eines Individuums mit Merkmal i (mit $ N_t := $ \#Individuen zur Zeit t mit Merkmal $ i $).\\
		$ d(i) + \sum_{j=1}^{n} c(i,j) \cdot n_t(i) $: & wie oben, nur mit $ n := $ \#Merkmale, und $ n_t(i) :=$ \#Individuen zur Zeit t mit Merkmal $ i $
	\end{tabular}\\
	
	Die letzte Darstellung der Todesrate ist praktischer f"ur die Betrachtung der Population. "Ahnlich k"onnen weitere Ereignisse zusammengefasst werden, so dass man z.B. eine Todesrate und eine arteigene Geburtenraten der Merkmale erstellen kann:

	\begin{itemize}
	 	\item Arteigene Geburtenrate des Merkmals i: 
	 	\[ b(i) \cdot (1 - \mu) + \left( b(i+1) \cdot \mathbbm{1}_{i>1} + b(i-1) \cdot \mathbbm{1}_{i>1} \right) \cdot \frac{\mu}{2} \] 
	 	\item Gesamte Todesrate des Merkmals i: 
	 	\[ d(x) + \sum_{i=1}^{n} c(x,x_i) \cdot n_t(x_i) \]
	\end{itemize}
	
	Die Simulation soll die Entwicklung der Merkmale und nicht die Ereignisse der Individuen darstellen, daher ist es unpraktisch diese zu betrachten. Alternativ werden die Ereignisse zu denen von Merkmalen zusammengefasst:
	\begin{itemize}
		\item Geburtenrate (Wachstumsrate) des Merkmals x: \[ B(x) = b(x) \cdot (1 - \mu) \cdot n_t(x) + (b(x+1)\cdot n_t(x+1) + b(x-1)\cdot n_t(x-1)) \cdot \frac{\mu}{2} \]
		(Notation deutlich machen)
		\item Todesrate des Merkmals x: 
		\[ D(x) = d(x) \cdot n_t(x) + n_t(x) \cdot \sum_{i=1}^{n} c(x,x_i) \cdot n_t(x_i) \]
	\end{itemize}
	
	
	
	
%	\newline
	\noindent\rule{\textwidth}{1pt}
	\begin{itemize}
		\item 
		Die Simulation soll die Entwicklung der Merkmale und nicht die Ereignisse der Individuen darstellen, daher ist es unpraktisch diese zu betrachten. Alternativ werden die Ereignisse zu denen von Merkmalen zusammengefasst:
		\begin{itemize}
			\item Geburtenrate (Wachstumsrate) des Merkmals x: \[ B(x) = b(x) \cdot (1 - \mu) \cdot n_t(x) + (b(x+1)\cdot n_t(x+1) + b(x-1)\cdot n_t(x-1)) \cdot \frac{\mu}{2} \]
			(Notation deutlich machen)
			\item Todesrate des Merkmals x: 
			\[ D(x) = d(x) \cdot n_t(x) + n_t(x) \cdot \sum_{i=1}^{n} c(x,x_i) \cdot n_t(x_i) \]
		\end{itemize}
		Das entspricht 2 exponentiellen Uhren pro Merkmal. Eine für Tod und eine für Geburt innerhalb des Merkmals.
		\item Zur praktischen Simulation ist eine Gesamtrate für das Eintreten eines Ereignisses praktischer. Auf diese Weise wird nur auf das eintreffen einer Uhr gewartet.
		\begin{itemize}
			\item Ereignisrate des Merkmals x (Trait Rate):
				\[ TR(x) = B(x) + D(x) \]
			\item Totale Ereignis Rate (Total Event Rate): 
			\[ TER = \sum_{x \in X} TR(x)\]
		\end{itemize}
		Mit der Totalen Ereignisrate gibt es jetzt eine Rate die es erlaubt eine Zufallsvariable für das Eintreffen einer Variable zu ziehen. Anschließend ist es nur noch erforderlich (mit der Ziehung zwei weiterer Zufallsvairablen) festzustellen welchem Merkmal welches Ereignis zukommt.
		\item Die Population ist ein Markov Sprungprozess der durch Zufallsvariablen
		\[ \nu_t = \sum_{i=1}^{N_t} \delta_{x_i}, \text{ mit } \int_X 1\nu_t(dx) = N_t \]
		beschrieben wird.\\
		Dabei gilt: \[ \nu_t \in M_F(X) = \left\{ \sum_{i=1}^{n} \delta_{x_i}, n \in \mathbb{N}, x_1, \dots, x_n \in X \right\} \]
		\item (Vielleicht nicht erwähnen?) Der Generator des so definierten BPDL-Prozess $ \nu_t $ ist:
		\begin{align}
			L_{\phi(\nu)} &= \int_{x} b(x)(1-\mu)[\phi(\nu + \delta_x) - \phi(\nu)]\nu(dx)\\
						&+ \int_{x}\int_{\mathbb{R}^d} b(x)(\mu)[\phi(\nu + \delta_{x+z}) - \phi(\nu)] m(x,dz) \nu(dx)\\
						&+ \int_{x} d(x)[\phi(\nu - \delta_x) - \phi(\nu)]\nu(dx)\\
						&+ \int_{x} \left( \int_{X} c(x,y) \nu(dy) \right) [\phi(\nu - \delta_x) - \phi(\nu)]\nu(dx)
		\end{align}
		mit $ \phi: M_F \to \mathbb{R} $
	\end{itemize}
	
	
	\subsection{Normalisierung}
	
	\subsection{Equilibrium}
	Man erkennt dass die Merkmale für $ K \to \infty $ eine Konvergenz gegen eine Funktion $ \xi_t = n_t \delta_x $ aufweist. Diese Funktion wiederum konvergiert gegen einen stabilen Zustand (im folgenden Equilibrium genannt), worin sich die Populationsgröße nicht mehr ändert. Diese sehen für monomorphe und dimorphe Populationen ohne Mutation folgendermaßen aus:
	\begin{itemize}
		\item Monomorphe Population\\
			\begin{align*}
			0 & = \dot{n} = (b(x) - d(x) - \bar{n}c(x,x))\bar{n}\\
			\bar{n} &= \frac{\left[ b(x)-d(x) \right]_+}{c(x,x)}
			\end{align*}
		\item Dimorphe Population\\
			\[ n_x = \frac{(b(x) - d(x))c(y,y)-(b(y)-d(y))c(x,y)}{c(y,y)c(x,x) - c(y,x)c(x,y)} \]
			\[ n = \frac{(b(y) - d(y))c(x,x)-(b(x)-d(x))c(y,x)}{c(y,y)c(x,x) - c(y,x)c(x,y)} \]
			oder $ (\bar{n}_x, 0)$, $ (0, \bar{n}_y)$ bzw. $ (0,0) $
	\end{itemize}
	Später im Kapitel TSS wird die Fitness Funktion eingeführt die näher erläutern kann wann welcher dimorphe stabile Zustand angenommen wird.
	In den Simulationen sind diese stabilen Zustände als gestrichelte Gerade angezeigt, gegen die Konvergenz stattfinden soll.

\section{Algorithmus}
	Der Simulation liegt ein Algorithmus zugrunde der einen Sprung des Markov Sprung Prozesses durchführt. Im Code wird dazu die "{}EvolutionStep()"{} Funktion aufgerufen. (Es gibt auch Jump statt Step, aber mehrere Schritte)
	\begin{algorithm}[H]
		\caption{EvolutionStep()}
		\begin{algorithmic}[1]
			\ENSURE{A full evolution Step happened}
			\STATE calculateEventRates();
			\STATE sampleEventTime();
			\STATE changeATrait();
		\end{algorithmic}
	\end{algorithm}
	
	Von dieser werden folgende Berechnungen angestoßen:
		
	\begin{algorithm}[H]
		\caption{EvolutionStep()}
		\begin{algorithmic}[1]
			\ENSURE{A full evolution Step happened}
			\STATE ---$>$calculateEventRates();
			\STATE calculateTotalDeathRates()
			\STATE calculateTotalBirthRates()
			\STATE calculateTotalEventRate()
			\STATE ---$>$sampleEventTime();
			\STATE sampleEventTime();
			\STATE ---$>$changeATrait();
			\STATE choseTraitToChange();
			\STATE choseEventType();
			\STATE executeEventTypeOnTrait();
		\end{algorithmic}
	\end{algorithm}
	
	Schließlich der Ablauf der tatsächlichen Berechnung:
	
	\begin{algorithm}[H]
		\caption{EvolutionStep()}
		\begin{algorithmic}[1]
			\ENSURE{A full evolution Step happened}
			\REQUIRE $ t, X = \{0,\dots, n-1\} $
			\STATE ---$>$calculateEventRates();
			\FOR{ $ x \in X $ }
				\STATE $  D(x) := n_t(x) \cdot \left( d(x) + \sum_{y \in X} c(x,y) \cdot n_t(y) \right) $
				\STATE $ B(x) := \underbrace{b(x) \cdot (1 - \mu) \cdot n_t(x)}_{arteigene}  $
				\IF{$ x > 0 $}
					\STATE $ B(x) += \underbrace{b(x-1)\cdot n_t(x-1)}_{Mutation Links} \cdot \frac{\mu}{2} $
				\ENDIF
				\IF{$ x < n-1 $}
					\STATE $ B(x) += \underbrace{b(x+1)\cdot n_t(x+1)}_{Mutation Rechts} \cdot \frac{\mu}{2} $
				\ENDIF
				\STATE $ TotalTraitRate(x) = B(x) + D(x) $
			\ENDFOR
			\STATE $ TotalEventRate := \sum_{x \in X} TotalTraitRate(x) $
			
			\STATE ---$>$sampleEventTime();
			\STATE sample $ Z \sim exp(TotalEventRate) $
			\STATE $ t += Z $
			
			\STATE ---$>$choseTraitToChange();
			\STATE sample $ Y \sim U(0,TotalEventRate) $
			\FOR{$ x \in X $}
				\IF{$ Y \le TotalTraitRate(x) $}
					\STATE $ ChosenTrait := x $
					\STATE break
				\ENDIF
				\STATE $ Y -= TotalTraitRate(x) $
			\ENDFOR
			
			\STATE ---$>$choseEventType();
			\STATE sample $ Y \sim U(0,TotalTraitRate(ChosenTrait)) $
			\IF{$ Y \le B(ChosenTrait) $}
				\STATE isBirht := true
			\ELSE
				\STATE isBirth := false
			\ENDIF
			
			\STATE ---$>$executeEventTypeOnTrait();
			\IF{isBirth}
				\STATE $ n_t(ChosenTrait) += 1 $
			\ELSE
				\IF{$ n_t(ChosenTrait) \ge 0 $}
					\STATE $ n_t(ChosenTrait) -= 1 $
				\ENDIF
			\ENDIF
		\end{algorithmic}
	\end{algorithm}
\section{Simulation}

\section{Korrektheit der Implementation}

\section{TSS Prozesse}

\section{Ausblick}

\begin{itemize}
	\item Weiteres Abbruchkriterium = Zeit : sehr einfach zu implementieren.
\end{itemize}
\bibliography{science1}

%\clearpage

\end{document}