\documentclass[11pt,a4paper,notitlepage]{article}
\usepackage[utf8]{inputenc}
\usepackage[ngerman]{babel}
\usepackage{amsmath}
\usepackage{amsthm}
\usepackage{amsfonts}
\usepackage{amssymb}
\usepackage{makeidx}
\usepackage[colorlinks=false, pdfborder={0 0 0}]{hyperref}

\title{Beweis 2.7 - Lipschitz Stetigkeit}

\newcommand{\tvec}[2]{\begin{pmatrix}#1\\#2\end{pmatrix}}

\begin{document}
\maketitle
Beh: Es gibt ein $ M_K < \infty $ so dass:
\[ \left| F\tvec{n_1^x}{n_1^y} - F\tvec{n_2^x}{n_2^y} \right| \overset{!}{<} M_K \left| \tvec{n_1^x}{n_1^y} - \tvec{n_2^x}{n_2^y} \right|\] 
mit $ n_{1,2}^{x,y} \in K $, $ K $ Kompakt, $ b(x), c(x,y), d(x) \forall x,y \in X \text{ endlich}$ und 
\begin{align}
	F\tvec{n^x}{n^y} = \tvec{n^x(b(x)-d(x)-c(x,x)n^x-c(x,y)n^y)}{n^y(b(y)-d(y)-c(y,y)n^y-c(y,x)n^x)}
	\label{nDGL}
\end{align}

\begin{proof}
	Betrachten wir zun"achst nur die erste Zeile:
	\begin{align*}
		 &&& M_K |n_1^x - n_2^x| >  |
		 (n_1^x - n_2^x)(b(x)-d(x))\\
		 &&&  - ((n_1^x)^2 - (n_2^x)^2)c(x,x) - ((n_1^y)^2 - (n_2^y)^2)c(x,y)|\\
		 \Leftrightarrow&&& M_K >  \left| b(x) - d(x)
		 - \frac{(n_1^x)^2 - (n_2^x)^2}{|n_1^x - n_2^x|}c(x,x) - \frac{(n_1^y)^2 - (n_2^y)^2}{|n_1^x - n_2^x|}c(x,y) \right|\\
		 \Leftrightarrow&&& \tilde{M}_K > \left| \frac{(n_1^x)^2 - (n_2^x)^2}{|n_1^x - n_2^x|} + \frac{(n_1^y)^2 - (n_2^y)^2}{|n_1^x - n_2^x|} \right|\\
		 \Leftrightarrow&&& \tilde{M}_K > \left| \underbrace{n_1^x + n_2^x}_{\text{beschr"ankt}} + \frac{(n_1^y)^2 - (n_2^y)^2}{|n_1^x - n_2^x|} \right|
	\end{align*}
	somit bleibt nur noch zu zeigen:
	\[ \left| \frac{(n_1^y)^2 - (n_2^y)^2}{|n_1^x - n_2^x|} \right| < \infty \]
	Dank der "Ahnlichkeit kann die zweite Zeile identisch umgeformt werden und es bleicht somit zu zeigen:
	\begin{align}
		\frac{(n_1^y)^2 - (n_2^y)^2}{|n_1^x - n_2^x|} < \infty\\
		\frac{(n_1^x)^2 - (n_2^x)^2}{|n_1^y - n_2^y|} < \infty
	\end{align}
	Addition der 1. und 2. Zeile ergibt:
	\begin{align*}
		&& \left| \frac{(n_1^y)^2 - (n_2^y)^2}{|n_1^x - n_2^x|} + 
		\frac{(n_1^x)^2 - (n_2^x)^2}{n_1^y - n_2^y} \right| &< \infty\\
		\overset{\text{gleicher Nenner}}{\Leftrightarrow} && \frac{|((n_1^y)^2 - (n_2^y)^2)(n_1^y - n_2^y)| + |((n_1^x)^2 - (n_2^x)^2)(n_1^x - n_2^x)|}{|n_1^x - n_2^x||n_1^y - n_2^y|} &< \infty\\
		\overset{\text{3. bin. Formel}}{\Leftrightarrow} && \frac{|(n_1^y - n_2^y)^2 (n_1^y + n_2^y)| + |(n_1^x - n_2^x)^2 (n_1^x + n_2^x)|}{|n_1^x - n_2^x||n_1^y - n_2^y|} &< \infty\\
		\overset{\text{spalten}}{\Leftrightarrow} && \left| \frac{(n_1^y - n_2^y) (n_1^y + n_2^y)}{n_1^x - n_2^x} \right| + \left| \frac{(n_1^x - n_2^x) (n_1^x + n_2^x)}{n_1^y - n_2^y} \right| &< \infty\\
		\overset{T}{\Leftrightarrow} && \left| \frac{n_1^y - n_2^y}{n_1^x - n_2^x} \right|  + \left| \frac{n_1^x - n_2^x}{n_1^y - n_2^y} \right|  \cdot \frac{n_1^x + n_2^x}{n_1^y + n_2^y}&< \infty\\
		\overset{\text{Konstante absch.}}{\Leftrightarrow} && \left| \frac{n_1^y - n_2^y}{n_1^x - n_2^x} \right|  + \left| \frac{n_1^x - n_2^x}{n_1^y - n_2^y} \right|  &< \infty\\
	\end{align*}
	Damit hat sich das Problem bestenfalls dazu ver"andert dass ich wissen muss dass sich $ n_1 $ und $ n_2 $ in beiden Traits gleich schnell ann"ahern m"ussen. Ab hier habe ich etwas Schwierigkeiten. Bin ich vielleicht auf dem falschen Weg oder habe etwas falsch gemacht?\\
	Ich versuche es derzeit argumentativ, aber bin etwas verwirrt durch die Natur von $ n_1, n_2 $. Sie l"osen also beide in jedem Punkt die Differentialgleichung (\ref{nDGL}). Aber wenn sie das tun, dann kann sich $ n_1^y $ und $ n_2^y $ nicht voneinander unterscheiden, wenn $ n_1^x = n_2^x $. Oder?
%	Problematisch ist es jedoch immer noch falls $ n_1^x \to n_2^x $ oder $ n_1^y \to n_2^y $. Jedoch l"asst sich diese Gleichung f"ur beiden F"alle so umformen, dass sie entsch"arft werden.
%	
%	\begin{align}
%		\frac{(n_1^y - n_2^y)^2 (n_1^y + n_2^y) + (n_1^x - n_2^x)^2 (n_1^x + n_2^x)}{n_1^x(n_1^y - n_2^y) - n_2^x(n_1^y + n_2^y)} &< \infty\\
%		\frac{(n_1^y - n_2^y)^2 (n_1^y + n_2^y) + (n_1^x - n_2^x)^2 (n_1^x + n_2^x)}{n_1^x(n_1^y - n_2^y) - n_2^x(n_1^y + n_2^y)} &< \infty
%	\end{align}
	
\end{proof}

\end{document}