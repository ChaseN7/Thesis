\documentclass[11pt,a4paper]{article}
\usepackage[utf8]{inputenc}
\usepackage[ngerman]{babel}
\usepackage{amsmath}
\usepackage{amsfonts}
\usepackage{amssymb}
\usepackage{makeidx}
\usepackage{graphicx}
\title{Theorem 2.1}
\begin{document}

\maketitle

Der Punkt 2.7 bedeutet f"ur unseren Fall, dass zwei L"osungen $ n, \tilde{n} $ des Differentialgleichungssystems die Eigenschaft haben, dass ihre Ableitungen 
$ \dot{n}, \dot{\tilde{n}} $ nicht viel gr"o"seren Abstand haben k"onnen als $ n, \tilde{n} $. In anderen Worten ist:
\[ \frac{\dot{n}(s) - \dot{\tilde{n}}(s)}{n(s) - \tilde{n}(s)} \le M_K\]
was einfach nur aussagt dass der Quotient $ \frac{\dot{n}(s) - \dot{\tilde{n}}(s)}{n(s) - \tilde{n}(s)} $ an jeder Stelle $ s $ nach oben beschr"ankt ist. \\
Ich vermute dass diese Bedingung f"ur unser Modell immer zutrifft, weil sie stets die selben stabilen Punkte haben zu denen sie konvergieren. Aber wahrscheinlich gibt es eine deutlich bessere Variante das zu begr"unden. Insbesondere m"usste ich meine Begr"undung erstmal nachweisen.

\end{document}