\documentclass[11pt, a4paper, german]{article}

\usepackage[german]{babel}
\usepackage{BA_Titelseite}


%Namen des Verfassers der Arbeit
\authornew{Boris Prochnau}

%Geburtsdatum des Verfassers
\geburtsdatum{22. Dezember 1989}
%Gebortsort des Verfassers
\geburtsort{Tartu}
%Datum der Abgabe der Arbeit
\date{\today}

%Name des Betreuers
% z.B.: Prof. Dr. Peter Koepke
\betreuer{Betreuer: Prof. Dr. Anton Bovier}
%Name des Instituts an dem der Betreuer der Arbeit t�tig ist.
%z.B.: Mathematisches Institut
\institut{Instituts f�r Angewandte Mathematik}
%Titel der Bachelorarbeit
\title{Simulation normalisierter BPDL Prozesse}
%Do not change!
\ausarbeitungstyp{Bachelorarbeit Mathematik}



\begin{document}

\maketitle
\tableofcontents

\clearpage


\section*{Ziel der Bachelorarbeit}
Das Ziel meiner Bachelorarbeit ist es eine Programm zu schreiben was den zeitlichen Verlauf von sogenannten BPDL ( von Bolker, Pacala, Dieckmann und Law) Prozessen graphisch darstellt.\\
Dabei ist das Interesse besonders bei normalisierten BPDL Prozessen und schlie�lich auch bei TSS (trait subsitution sequence) Prozessen.\\
Dabei reicht es nicht einfach eine Implementierung zu machen, denn durch die Zufallseigenschaft l�sst sich nicht so einfach die Korrektheit verifizieren. Gerade bei TSS Prozessen ist die Anzahl der Mutationen und deren Abst�nde und Invasionschancen entscheidend.\\
Zu diesem Zweck wird ein Teil der Arbeit diese Problemstellung behandeln.

\section*{Model}
Das verwendete Model lehnt sich an das Model aus [...] an. Jedoch wurde f�r meine Simulation einige kleine �nderungen gew�nscht.\\
Grundger�st:
	\begin{itemize}
		\item Jedes Individuum hat ein Merkmal $ x \in X $. \\
		Der Einfachheit halber sei X eine Indexmenge: $ X = \{1,\dots, n\} $ repr�sentativ f�r eine Durchz�hlung der Merkmale.
		\item Jedes Individuum kann sich asexuell fortpflanzen oder sterben
		\item Tod und Fortpflanzung sind Ereignisse deren erstes Eintreffen durch sogenannte exponentielle Uhren beschrieben werden k�nnen. Diese Uhren haben exponentiell verteilte Weckzeiten. Durch die Ged�chtnislosigkeit der Exponentialverteilung, k�nnen die Uhren nach dem klingeln der ersten neu gestellt werden. 
	\end{itemize}
	Gleich wird klar dass diese Eigenschaft entscheiden ist um die Sichtweise von der Ebene des Individuums auf die gesamte Population zu heben.\\
	Diese Todes und Fortpflanzungs - Ereignisse eines Individuums haben feste Raten die das Merkmal des Individuums auszeichnen.
	\begin{itemize}
		\item b(x): Geburtenraten durch ein Individuum mit Merkmal x
		\item d(x): nat�rliche Todesrate
		\item c(x,y): Todesrate durch Wettbewerb zwischen Individuen mit Merkmal x und y.
		\item $ \mu $: Mutationswahrscheinlichkeit "{}auf die Nachbarn"{} mit je $ \frac{\mu}{2} $ pro Nachbar. 
	\end{itemize}
	Durch Superposition lassen sich die beiden Todesraten zu einer gemeinsamen Todesrate zusammenfassen oder die Geburtenrate z.B. zu einer intrinsischen Geburtenrate auftrennt.
	\begin{itemize}
		\item intrinsische Geburtenrate: $ b(x) \cdot (1 - \mu) $
	 	\item Todesrate: $ d(x) + \sum_{i=1}^{N_t} c(x,x_i) $, $ N_t = \#Individuen$ zum Zeitpunkt t, und \underline{$ x_i $ das Merkmal des i-ten Individuums}.
	 	\item ODER Todesrate: $ d(x) + \sum_{i=1}^{n} c(x,x_i) \cdot n_t(x_i) $, $ n = \#Merkmale$, und $ n_t(x_i) = \#Individuen $ mit \underline{Merkmal $ x_i $} zur Zeit t.
	\end{itemize}

\section*{Algorithmus}

\section*{Simulation}

\section*{Korrektheit der Implementation}

\section*{TSS Prozesse}

%\clearpage

\end{document}
