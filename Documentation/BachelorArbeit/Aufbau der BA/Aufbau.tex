
\documentclass[11pt, a4paper, german]{article}
\usepackage[utf8]{inputenc}
\usepackage{amsmath}
\usepackage{amsfonts}
\usepackage{amssymb}
\usepackage{algorithm}
\usepackage{algorithmic}
\usepackage{makeidx}
\usepackage{graphicx}
\usepackage{tabularx}
\usepackage{bbm}
\usepackage[ngerman]{babel}
\usepackage{BA_Titelseite}
%\bibliographystyle{unsrt}

%Namen des Verfassers der Arbeit
\authornew{Boris Prochnau}

%Geburtsdatum des Verfassers
\geburtsdatum{22. Dezember 1989}
%Gebortsort des Verfassers
\geburtsort{Tartu}
%Datum der Abgabe der Arbeit
\date{\today}

%Name des Betreuers
% z.B.: Prof. Dr. Peter Koepke
\betreuer{Betreuer: Prof. Dr. Anton Bovier}
%Name des Instituts an dem der Betreuer der Arbeit tätig ist.
%z.B.: Mathematisches Institut
\institut{Institut für Angewandte Mathematik}
%Titel der Bachelorarbeit
\title{Simulation normalisierter BPDL Prozesse}
%Do not change!
\ausarbeitungstyp{Bachelorarbeit Mathematik}



\begin{document}

\maketitle
\tableofcontents

\clearpage


\section{Einleitung}
Hier steht eine Einleitung bestehend aus:\\
Ziel dieser Bachelorarbeit...\\


\clearpage
\section{Modell}

	\subsection{Grundlagen}
	
	\subsection{BPDL Prozess}
	
		
	
	
\clearpage
\section{Normalisierung und Eigenschaften des BPDL Prozesses}

\subsection{Normalisierung des BPDL Prozesses}

	\subsection{Fitness}
	
	\subsection{Equilibrium}


\clearpage

\noindent\rule{\textwidth}{2pt}
\begin{center}
	$ <<< \quad \downarrow $ In Arbeit $ \downarrow \quad >>> $
\end{center}
	
\section{Algorithmus zur Simulation eines normalisierten BPDL Prozesses}
	\subsection{Implementierung}
	
	\subsection{Pseudocode}

	\subsection{Optimierung f"ur viele Merkmale}
	
\clearpage

\section{Simulation und Programmablauf}

	\subsection{Aufgabenteilung und Flexibilität}
	
	\subsection{Layout}


\clearpage
\section{Verhaltenstest - Korrektheit der Implementation}
	\subsection{Unit Tests der Algorithmusmodule}

\clearpage
\section{TSS Prozesse}

	\subsection{Algorithmus}
	
\clearpage
\section{Ausblick}

\clearpage
%\bibliography{science1}

\end{document}